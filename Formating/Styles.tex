%%%%%%%%%%%%%%%%%%%%%%%%% Notes %%%%%%%%%%%%%%%%%%%%%%%%%%%%%%
% Author: Øistein Jelmert Skjolddal
% Loosely based on the moderncv template

% This section contains specific element styles.
%%%%%%%%%%%%%%%%%%%%%%%%%%%%%%%%%%%%%%%%%%%%%%%%%%%%%%%%%%%%%%


%\ifx \Macro1 \Macro2 %<tex commands>
%\else %<tex commands>
%\fi


%-------------- CV entry style section --------------------

\newcommand*{\cvEntry}[6][.25em]{ % 6 elements on the form: 
        %{\cvEntry{2dn bracket}{3rd bracket}{4th bracket}{5th bracket}{6th bracket}}
    \begin{tabular}{l} % First element on the first line, contained in the 2nd {} 
        {\bfseries #2} % Bold the content of the 2nd {}
    \end{tabular}
    \hfill % Move the following element to the right
    \begin{tabular}{l} % Second element on the first line, , contained in the 3rd {}
        {\bfseries #3} % Bold the content of the 3rd {}
    \end{tabular} \\ % Newline
    \begin{tabular}{l} % First element on the new line 3rd element, contained in the 4th {}
        {\itshape #4} % Write the content of the 4th {} in italic
    \end{tabular}
    \hfill % Move the following element to the right
    \begin{tabular}{l} % Second element on the new line. 4th element, contained in the 5th {}
        {\itshape #5} % Write the content of the 5th {} in italic
    \end{tabular}

    % https://en.wikibooks.org/wiki/TeX/ifx
    \ifx&#6& % If nothing in the 7th bracket, the nothing 
    \else{
    % http://www.sascha-frank.com/latex-minipage.html
        %\begin{minipage}{\maincolumnwidth}
        \begin{minipage}{170mm}
            \small#6 % With the content of element 7
        \end{minipage}}
    \fi % end mini, end if (\fi)
    \par \addvspace{#1}} % New paragraph, put each new element with in the space of the cvEntry (bracket 1)

%-------------- Programming Languages --------------------    
\newcommand*{\cvLanguage}[4][.25em]{ % 3 elements on the form: 
        %{\cvLanguage{2dn bracket}{3rd bracket}{4th bracket}}
    \begin{tabular}{p{6cm}} % First element on the first line, contained in the 2nd {} 
        {\bfseries #2} % Bold the content of the 2nd {}
    \end{tabular}
    \begin{tabular}{l} % First element on the first line, contained in the 2nd {} 
    {\bfseries #3} % Bold the content of the 3rd {}
    \end{tabular}
    \hfill % Move the following element to the right
    \begin{tabular}{l} % Second element on the first line, , contained in the 3rd {}
        {\bfseries #4} % Bold the content of the 4th {}
    \end{tabular} %\\ % Newline
    \par \addvspace{#1}} % New paragraph, put each new element with in the space of the cvEntry (bracket 1)

%-------------- Programming Languages headlines --------------------    
\newcommand*{\LanguageHeader}[4][.25em]{ % 3 elements on the form: 
     %{\cvLanguage{2dn bracket}{3rd bracket}{4th bracket}}
    \begin{tabular}{l} % First element on the first line, contained in the 2nd {} 
        {\bfseries #2} % Bold the content of the 2nd {}
    \end{tabular}
    \hfill % Move the following element to the right
    \begin{tabular}{l} % First element on the first line, contained in the 2nd {} 
        {\bfseries #4} % Bold the content of the 3rd {}
    \end{tabular}
    \par \addvspace{#1}} % New paragraph, put each new element with in the space of the cvEntry (bracket 1)

%-------------- Programming Languages 2 --------------------    
\newcommand*{\Language}[4][.25em]{ % 3 elements on the form: 
     %{\cvLanguage{2dn bracket}{3rd bracket}{4th bracket}}
    \begin{tabular}{l} % First element on the first line, contained in the 2nd {} 
        {#2} % The content of the 2nd {}
    \end{tabular}
    \hfill % Move the following element to the right
    \begin{tabular}{l} % First element on the first line, contained in the 2nd {} 
        {#4} % The content of the 3rd {}
    \end{tabular}
    \par \addvspace{#1}} % New paragraph, put each new element with in the space of the cvEntry (bracket 1)
    
 